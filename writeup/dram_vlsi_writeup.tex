\documentclass[11pt]{article}
\usepackage{amsmath}
\usepackage{geometry}
\geometry{margin=1in}

\title{16$\times$4 DRAM Project Report}
\author{Adithya Selvakumar \& Krishna Karthikeya Chemudupati}
\date{}

\begin{document}
\maketitle

\newpage

\tableofcontents
\newpage

% -----------------------------------------
\section{Memory Description}

\subsection{Text Description of Memory Operation}
% 1.1

The proposed memory is a 16$\times$4 dynamic random-access array in which each bit is stored as charge on a dedicated capacitor accessed through a single NMOS switch. Because the stored information is represented by an analog voltage rather than a bistable digital state, every memory operation must explicitly manage the charge on the storage capacitor and the large distributed capacitance of the bitline. At the system level, the memory supports synchronous read and write access to any of the sixteen 4-bit words using a shared periphery consisting of a row decoder, bitline precharge network, write drivers, and dynamic sense amplifiers. All operations follow a well-defined sequence controlled by the global clock to ensure that the bitline reference level, access timing, and sense-amplifier activation are coordinated precisely.

Before any access, each bitline is precharged to a known reference voltage, typically $V_{\mathrm{DD}}/2$, to establish a balanced initial condition for charge sharing. This reference level minimizes read disturb, reduces voltage swing requirements, and ensures that both a ``stored zero'' and a ``stored one'' shift the bitline in opposite directions by comparable magnitudes. Once precharge is complete, the selected wordline is asserted. This connects each storage capacitor in the addressed row to its respective bitline, initiating a small charge redistribution governed by the ratio between the cell capacitance $C_{S}$ and the total bitline capacitance $C_{\mathrm{BL}}$. The resulting voltage perturbation,
\[
\Delta V = \frac{C_{S}}{C_{S}+C_{\mathrm{BL}}} \left( V_{\mathrm{cell}} - V_{\mathrm{PRE}} \right),
\]
is typically only a few hundred millivolts, and therefore insufficient to be interpreted directly as a logic level. A dynamic sense amplifier resolves this small deviation into a full-rail digital value by temporarily isolating the bitline and regeneratively amplifying the sign of the perturbation. This same regeneration step also rewrites the correct level back into the cell, ensuring that the destructive read is paired with immediate restoration.

Write operations bypass the charge-sharing mechanism by directly forcing full-swing logic levels onto the bitlines using dedicated write drivers. When the wordline is raised, the bitline voltage overwrites the charge on the storage capacitor through the access transistor. Because the NMOS access device imposes a threshold drop when writing a logical \texttt{1}, the write driver is sized and biased such that the cell capacitor is still charged to a level distinguishable from a stored \texttt{0} during subsequent reads. After the wordline returns low, the capacitor retains its updated voltage until it naturally leaks.

Due to leakage through the access transistor and junction parasitics, the stored charge decays over time, making periodic refreshing essential for correct operation. The refresh controller cycles through all sixteen rows, performing a controlled read-and-restore sequence identical to a normal read but without driving data externally. This ensures that each capacitor is periodically replenished before its voltage degrades below the sense margin. Together, these operations define a fully synchronous DRAM system in which precise timing of precharge, activation, sensing, and restoration enables reliable random access despite the analog and transient nature of charge-based storage.

\newpage

\subsection{Description of Timing Requirements}
% 1.2

Reliable operation of the DRAM array is governed by a sequence of tightly constrained timing phases that control how charge is prepared, transferred, sensed, and restored. Because the stored data is represented as charge on a small capacitor rather than a stable latch, each phase must guarantee proper settling of analog voltages before the next action occurs. The minimum allowable clock period for the memory is thus defined by the slowest of these phases, and any deviation in their ordering or duration risks incorrect sensing, incomplete restoration, or accelerated data loss.

The timing sequence begins with bitline precharge, during which all columns are driven to a uniform reference level $V_{\mathrm{PRE}} = V_{\mathrm{DD}}/2$. This must fully accommodate the large bitline capacitance $C_{\mathrm{BL}}$, ensuring that every bitline settles within a narrow tolerance around the reference. Since $C_{\mathrm{BL}} \gg C_{S}$, insufficient precharge time directly reduces sense margin and increases the risk of ambiguous reads. Only once the bitlines reach steady state is the precharge network disabled, placing a firm lower bound on precharge duration dictated by the driver strength and bitline RC.

After precharge, the selected wordline is asserted. Wordline activation timing is critical because the access transistor connects the small cell capacitor to the large bitline, initiating charge sharing. The activation time must be long enough for the bitline perturbation
\[
\Delta V = \frac{C_{S}}{C_{S}+C_{\mathrm{BL}}} \left( V_{\mathrm{cell}} - V_{\mathrm{PRE}} \right)
\]
to reach a detectable level. If the wordline is deasserted too early, $\Delta V$ may fall within the sense amplifier’s metastability region; if held too long, leakage and parasitic conduction begin to counteract the perturbation. The required pulse width therefore depends on the access transistor’s on-resistance, wordline RC delay, and the cell-to-bitline capacitance ratio.

The sense amplifier enable (SE) must be carefully aligned after the charge-sharing interval. Activating the sense amplifier too early risks incorrect polarity resolution, while enabling it too late allows the bitline to drift back toward $V_{\mathrm{PRE}}$. Once SE is asserted, the amplifier regeneratively drives the bitline to a full-rail level and simultaneously restores the cell’s charge. The sense window must remain open long enough for complete regeneration, since partial restoration reduces retention time and can corrupt subsequent reads.

Write operations impose their own timing requirements. The write driver must establish the correct bitline level before wordline activation, and the wordline must remain high long enough for the cell capacitor to charge or discharge through the NMOS access device. Writing a logical ``1'' is particularly sensitive because of the $V_{\mathrm{Tn}}$ drop across the access transistor, requiring a settling interval sufficient to leave the capacitor with a distinguishable stored high level. Wordline deassertion must occur only after the capacitor reaches its intended voltage.

Finally, periodic refresh introduces a global timing constraint. Every row must be revisited within the allowed retention interval using a read-and-restore operation identical to a normal read but hidden from external interfaces. The refresh window is determined by leakage paths and the minimum resolvable $\Delta V$ of the sense amplifier. The clock must therefore allocate enough idle cycles to refresh all sixteen rows, establishing an upper bound on usable memory bandwidth.



\newpage

\subsection{Optimization and Design Tradeoffs}
% 1.3



\newpage

% -----------------------------------------
\section{Schematics with Size Annotations}
% 2

\subsection{Baseline Design}

In this section, we present the baseline DRAM-based 16$\times$4 memory design using schematic views from Cadence, annotated with device dimensions for all transistors. The baseline implementation prioritizes functional correctness and clear hierarchy over aggressive optimization of area, power, or delay. Each schematic is organized according to the design tiers defined in our architecture: primitive logic gates (Tier~0), core analog and memory blocks (Tier~1), row and column infrastructure (Tier~2), control logic (Tier~3), verification environment (Tier~4), and full-array integration (Tier~5). For each circuit, we will eventually include a figure reference and a brief description of sizing choices, focusing on transistor widths, lengths, and any deliberate upsizing for drive strength or noise margin.

\subsubsection{Tier 0: Logic Primitives}

\noindent\underline{INV} \\
% Include schematic of baseline inverter with W/L annotations for pull-up and pull-down devices.
% Discuss chosen ratio and its use as a reference cell for other gates.



\newpage

\noindent\underline{NAND2} \\
% Include schematic of 2-input NAND gate sized relative to INV.
% Note series NMOS stack sizing and any upsizing for critical paths (e.g., decoder).



\newpage

\noindent\underline{NOR2} \\
% Include schematic of 2-input NOR gate with size annotations.
% Comment on parallel NMOS / series PMOS sizing, especially where used in predecoding.



\newpage

\subsubsection{Tier 1: Core Memory and Analog Blocks}

\noindent\underline{1T1C Bitcell} \\
% Show 1T1C DRAM cell schematic with access NMOS W/L and explicit storage capacitor value.
% Note chosen CS and its relationship to CBL and sense margin.

For the baseline 1T1C DRAM implementation, the storage capacitance $C_{S}$ and bitline capacitance $C_{\mathrm{BL}}$ were chosen to ensure correct functional behavior while preserving the qualitative characteristics of dynamic charge-based memory operation. Since all transistors in the baseline design use the minimum geometry of 120\,nm/45\,nm, the access device presents a relatively high on-resistance, and the charge-sharing dynamics must be driven primarily by the relative magnitudes of the explicit capacitors. We selected a storage capacitor of $C_{S}=20\,\mathrm{fF}$, representative of a small integrated capacitor in deeply-scaled technologies, and a bitline capacitance of $C_{\mathrm{BL}}=80\,\mathrm{fF}$, approximately four times larger to reflect the fact that real bitlines accumulate diffusion, wiring, and access-transistor capacitances from many cells. With these values, the read perturbation generated during charge sharing is
\[
\Delta V = \frac{C_{S}}{C_{S}+C_{\mathrm{BL}}}\left(V_{\mathrm{cell}} - V_{\mathrm{PRE}}\right) \approx 0.2 \times 0.6\,\mathrm{V} = 120\,\mathrm{mV},
\]
which produces a clear, measurable deviation from the precharge level while remaining within the small-signal regime expected of DRAM sensing. This $\Delta V$ is sufficiently large to be resolved by the baseline dynamic sense amplifier without requiring aggressive device sizing or timing precision, yet small enough to preserve the qualitative behavior of realistic DRAM columns where $C_{\mathrm{BL}} \gg C_{S}$. These capacitor values also ensure that the RC time constant of the charge-sharing process remains manageable despite the minimum-sized access transistor, enabling full read and restore behavior within relaxed wordline pulse widths during functional verification. As a result, the chosen $C_{S}$ and $C_{\mathrm{BL}}$ provide a robust and physically meaningful baseline environment in which the full read, write, and refresh mechanisms can be validated prior to any optimization of area, power, or delay.

\newpage

\noindent\underline{Precharge Circuit} \\
% Show bitline precharge network that drives BL to VDD/2 (or chosen reference).
% Annotate precharge transistor sizes and any equalization devices between bitlines.



\newpage

\noindent\underline{Dynamic Sense Amplifier} \\
% Show dynamic sense amp schematic (cross-coupled structure, isolation devices, SE input).
% Annotate input pair and load device sizes; explain relative sizing for gain and speed.



\newpage

\noindent\underline{Write Driver} \\
% Show write driver that forces BL to logic 0/1 during writes.
% Include W/L for pull-up and pull-down devices and any series gating controlled by WE.



\newpage

\subsubsection{Tier 2: Row and Column Infrastructure}

\noindent\underline{Wordline Driver} \\
% Show buffer chain or driver inverter for WL, sized to drive total wordline capacitance.
% Annotate each stage and indicate fanout / tapering if used.



\newpage

\noindent\underline{Row Decoder (4-to-16)} \\
% Show hierarchical 4-to-16 decoder schematic using Tier 0 gates.
% Annotate transistor sizes for predecode and final wordline-select stages.



\newpage

\subsubsection{Tier 3: Control Logic}

\noindent\underline{Read/Write FSM} \\
% Show state machine or control logic that sequences precharge, activate, sense, and write.
% Note that logic is built from Tier 0 gates; annotate key transistor sizes where timing-critical.



\newpage

\noindent\underline{Refresh FSM} \\
% Show refresh controller that iterates through row addresses within retention interval.
% Include any counters or control outputs with gate-level implementation.



\newpage

\noindent\underline{Precharge Control} \\
% Show gating logic that generates precharge enable and equalization signals.
% Annotate devices driving precharge transistors and their sizing.



\newpage

\subsubsection{Tier 4: Verification and Test Infrastructure}

\noindent\underline{Testbench} \\
% Top-level schematic used only for simulation, instantiating the DRAM module and sources.
% Include controlled voltage/current sources and measurement points.



\newpage

\noindent\underline{Stimulus} \\
% Show pattern-generation logic or piecewise sources for address, data, WE, and control signals.
% Document any parameterized sources used for read/write sequences and stress tests.



\newpage

\noindent\underline{CLK Block} \\
% Show clock-generation schematic (or ideal source representation) used in baseline simulations.
% Indicate period, duty cycle, and any derived phase signals (e.g., SE, precharge).



\newpage

\subsubsection{Tier 5: Full Array Integration}

\noindent\underline{16$\times$4 Array (Hierarchical)} \\
% Show hierarchical schematic instantiating 16 wordlines by 4 bitlines of 1T1C cells.
% Include connections to WL drivers, bitlines, sense amps, precharge circuits, and decoders.



\newpage

\noindent\underline{Top-Level DRAM Module} \\
% Show final top-level block exposing address, data in/out, WE, CLK, and any control/status pins.
% Annotate how lower-tier blocks are interconnected and where timing-critical paths originate.



\newpage

\subsection{Optimized Design}

% This subsection will be populated after baseline functionality is verified and
% FOM-driven optimizations (area, power, delay) are performed.
% We will present only those schematics that differ from the baseline and highlight
% the specific sizing or structural changes that reduce the overall FOM.



\newpage

% Include all block-level schematics and test schematics

% -----------------------------------------
\section{Reasonable Size}

\subsection{Memory Cell Area}
% 3.1



\newpage

\subsection{Other Area Considerations}
% 3.2



\newpage

\subsection{Description of Sizing Rationale with Support}
% 3.3



\newpage

% -----------------------------------------
\section{Design Functionality}

\subsection{Memory Design Validation Description}
% 4.1



\newpage

\subsection{Read Operation Demonstration and Explanation}
% 4.2



\newpage

\subsection{Read Works}
% 4.3



\newpage

\subsection{Write Operation Demonstration and Explanation}
% 4.4



\newpage

\subsection{Write Works}
% 4.5



\newpage

\subsection{Simulated Read/Write to Different Addresses}
% 4.6



\newpage

\subsection{Simulated Read/Write to Same Address}
% 4.7



\newpage

\subsection{Clean, Robust Design}
% 4.8



\newpage

% -----------------------------------------
\section{Reasonable Performance}

\subsection{Precharge Bitlines}
% 5.1



\newpage

\subsection{Write Driver Driving Bitlines}
% 5.2



\newpage

\subsection{Address to Wordline Select}
% 5.3



\newpage

\subsection{Cell Driving Bitline During Read}
% 5.4



\newpage

% -----------------------------------------
\section{Design Metrics with Support}
% 6



\newpage

% FOM, delay, power, area evidence

\end{document}
